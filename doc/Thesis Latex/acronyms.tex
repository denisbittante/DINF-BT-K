


\newglossaryentry{nfa}{
  name=NFA,
  description={Nicht-Funktionale Anforderungen an das System die man als Ergebnis beobachten kann, wenn das System in Betrieb ist}}
\newglossaryentry{soa}
{
  name=SOA,
  description={Service Oriented Architecture, beschreibt den Grundgedanken von verteilten Systemen herunter gebrochen auf einzelne Services}
}


\newglossaryentry{osre}
{
  name=OSRE,
  description={Open Source Reporting Engine, Framework für das Verarbeiten und Erstellen von Reports}
}

\newglossaryentry{post}{
name=POST, 
description={Eine der HTTP-Anfragemethoden}
}

\newglossaryentry{ram}{
name=RAM, 
description={Random-Access Memory bezeichnet den Arbeitsspeicher bei Computern}
}


\newglossaryentry{json}{
name=JSON, 
description={JavaScript Object Notation, ein Datenaustauschformat}
}

\newglossaryentry{swap}
{name =SWAP,
description= {ist die Auslagerungsdatei die genutzt wird um den physischen Arbeitsspeicher zu erweitern} }

\newglossaryentry{dyno}
{name =Dyno,
description= {Dyno ist ein Webserver bei der PaaS Heroku} }








\newacronym{pdf}{PDF}{Portable Document Format }
\newacronym{kpi}{KPI}{Key Performance Indicator}
\newacronym{kmu}{KMU}{Kleine und mittlere Unternehmen}
\newacronym{foss}{FOSS}{Free Open Source Software}
\newacronym{rest}{REST}{Representational State Transfer}
\newacronym{http}{HTTP}{Hypertext Transfer Protocol}
\newacronym{paas}{PaaS}{Platform as a Service }
\newacronym{rss}{RSS}{resident set size}
\newacronym{cli}{CLI}{command-line interface}
\newacronym{cpu}{CPU}{central processing unit}
\newacronym{vu}{VU}{virtuelle User}
\newacronym{api}{API}{application programming interface}
\newacronym{iaas}{IaaS}{Infrastructure as a Service}