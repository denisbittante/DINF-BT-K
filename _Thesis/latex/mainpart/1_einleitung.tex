\documentclass[main.tex]{subfiles}
\begin{document}

%%%
%%%  Einleitung 
%%%

%% Korrektur eingeflossen

\chapter{Einleitung}





Die vorliegende Arbeit beschäftigt sich mit der Evaluation von \gls{osre}. Die Frage nach der leistungsfähigsten \gls{osre} ist von besonderem Interesse, weil immer häufiger \gls{soa} im Einsatz sind.  Diese Architekturen befassen sich mit verteilten Systemen, die sich gezielt auf eine Aufgabe fokussieren. Dies ist in dieser Arbeit das Generieren von Reports im PDF-Format.




Die Evaluation von OSRE ist von Interesse, da sich die reine Evaluation der Performance auf die Infrastruktur auswirkt. Je effizienter und schneller die \gls{osre} die Anforderungen erfüllen, desto schwächer bzw. günstiger kann die Infrastruktur ausfallen, was auch bedeutet, dass diese wirtschaftlich gesehen schneller rentieren werden. Im aktuellen Markt zeigt sich der Trend zu Containern und \acrfull{paas}, welche horizontal skalierbar sind und wo aus Sicht der Hardware keine Grenzen gesetzt sind.  Performance- und Lasttests  können in diesem Anwendungsfall (Generieren der PDFs) aufzeigen, wie sich ein solcher Webserver verhalten würde. Diese Services können unter Umständen Ausfälle und Leistungsengpässe einzelner Container oder Knoten im Verbund vorhersagen. Damit lassen sich Infrastrukturen planen und administrieren, um diese z.B. in Bezug auf die aktuelle User-Aktivität auszulegen, auch wenn diese auf einer \acrshort{paas} oder \acrfull{iaas} betrieben werden sollen. 


\end{document}
