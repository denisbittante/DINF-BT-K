\documentclass[main.tex]{subfiles}
\begin{document}
\chapter{Diskussion} 

Welche Open Source Reporting Engine bietet die beste Leistung, um performante Services für skalierbare Webanwendungen zu implementieren?
Welche bietet den höchsten Durchsatz pro Sekunde?
Welche benötigt die wenigsten Ressourcen?

Zu diesen Fragen wurde in dieser Arbeit eine oder mehrere Antworten gesucht. Es konnten zum Teil Hinweise gefunden werden, aber nur wenn einzelne Test-Szenarien selektiv betrachtet werden. Je nach Anwendungsfall haben sich die \acrshort{osre}s anders verhalten. Auch konnten keine perfekten oder vollständige Layouts mit allen Prototypen umgesetzt werden. Dadurch entstanden viele kleine Diskrepanzen zwischen den \acrshort{osre}s, was sie nicht ganz vergleichbar werden lies. Trotzdem konnten Rückschlüsse auf die verschiedenen Verhalten gezogen werden.  

Wie in der Einführung erwähnt, sind heutzutage viele \acrfull{soa} Applikationen auf dem Markt und angesichts der Resultate kann gesagt werden, dass sich eine Performance-Evaluation für alle Services lohnt. Eine Aufgabe wie PDFs erstellen, kann wie gezeigt, unterschiedliche Resultate hervorbringen von einem langsamen bis schnellem Service oder von einem robusten zu einem Last empfindlichem Service. Natürlich wird in der Praxis aus Kosten oder Zeitgründen solche Test oft nicht durchgeführt, doch viele Beispiele zeigen, dass sich dieser Aufwand lohnen würde \cite[vgl.~Kap.1]{erinle_2013}. Auch wenn es bedeutet in Personal und Infrastruktur zu investieren.   
\section{Ergebnisse}
Die Test haben eine Reihe eher eindeutiger Ergebnisse geliefert. Wie z.B., dass Latenzzeit und Durchsatz sinkt, sobald die Last wie Requestgrösse oder Anzahl \acrlong{vu} steigen. Ebenfalls ist die Interpretation von Memory-Verbrauch und Antwortzeiten in Einklang zu bringen. Die Performance von JasperReports zeigt auf, wie eine RAM-intensive Applikation auf einer \acrlong{paas} aussehen kann. Ein weiteres Ergebnis, dass sich aus den Memory-Aufzeichenen erschliesst, sind Memory-Blasen, die auch im Zusammenhang mit JasperReports aufgezeichnet wurden. Das langsame Abbauen der Prozessen im Memory oder das Festhalten an Ressourcen, kann einen Webserver über eine längere Zeit zum Absturz bringen oder die Antwortszeit so sehr verlangsamen, dass die User schlecht auf dem System weiter arbeiten können. Diese Ergebnisse schneiden sich mit den Ergebnissen aus der Praxis. Ebenfalls zeigten diese Resultate auf, wie ein nicht reagierender Service nicht gleich abgestürzter Service bedeutet. Aus den Verfügbarkeits-Analysen hätten viele User die Services von JasperReports als nicht reagierend eingestuft. 
\section{Ungereimtheiten}
Es gab auch Ungereimtheiten, die aus den Test hervor kamen, wie das Szenario drei. Die Performance von iText und JasperReports haben in den ersten Szenarien gute Antwortzeiten und Durchsatz sowie eine gute Handhabung der Ressourcen gezeigt. Selten haben diese den Webserver gefordert, ausser beim letzten Szenario. Mit JasperReports hingegen, haben sich keine grössere Ausreisser manifestiert. Es hat im gleichen Rhythmus die PDFs generiert. Einzig die Verarbeitung von Bildern war im letztem Szenario hinzugekommen und hob sich von den vorgehenden Szenarien ab. Diese Anforderung hat aufgezeigen können, dass einerseits die generierten PDFs, unter Berücksichtigung ihrer Grösse, sich stark untereinander unterscheiden, andererseits dabei einen Hinweis liefert warum die tiefe Performanz von JasperReports zu beobachten war. Die Ungereimtheit besteht hierbei, dass der Memory-Verbrauch bei iText zwar steigt, aber die Auslastung des CPU auf unter zwei Prozessen sinkt. iText erreicht diese Tiefe in keinem anderen Szenario.  Apache PDFBox erreicht dabei einen hohen Durchsatz in Bytes und scheint, als ob das generieren keine Mühe machen würde. Das Memory ist nicht an die RSS-Memory-Quota angekommen, sowie die CPU-Last ist nur gering höher, als in den vorgehenden Szenarien.  Das Ergebnis wird wie folgt beurteilt. Es wird der Zugriff auf die Bilder, also das I/O als das Bottleneck zu vermuten sein und damit den Durchsatz herabsetzen. Diese Annahme würde auch die tiefe Performance von JasperReports über alle Szenarien hinweg erklären da diese seine Templates ebenfalls, gleich wie die Bilder, laden muss.
\section{Schlussfolgerungen}
Als wichtige Schlussfolgerungen aus dieser Arbeit ist folgendes einzubeziehen. Das Testen der Performance, egal welcher Service, ist im Bereich Microservices oder \acrshort{soa} eine wichtige Qualitätsprüfung, die Kosten und Zeit sparen kann. Die Performance von \acrshort{osre}s hängt stark von den Anwendungsfall ab. Es lassen sich pauschal nur Hinweise zur Performance geben. Die Last kann einen Service stark beinflussen. Dabei spielen alle Messgrössen zusammen und können nur gemeinsam eine Antwort zur Performance geben.  


\end{document}